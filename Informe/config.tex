\input{.command.tex}
% En el siguiente archivo se configuran las variables del trabajo práctico
%% \providecommand es similar a \newcommand, salvo que el primero ante un 
%% conflicto en la compilación, es ignorado.

% Al comienzo de un TP se debe modificar los argumentos de los comandos

\providecommand{\myTitle}{TRABAJO PRÁCTICO 1}
\providecommand{\mySubtitle}{Dinámica de movimiento de un vehículo}

\providecommand{\mySubject}{Procesamiento de Señales II (86.52)}
\providecommand{\myKeywords}{UBA, Ingeniería, PS2}

\providecommand{\myAuthorSurname}{Manso}
\providecommand{\myTimePeriod}{Año 2018 - 2\textsuperscript{do} Cuatrimestre}

% No es necesario modificar este %%%%%%%%%%%%%%
\providecommand{\myHeaderLogo}{header_fiuba}
%%%%%%%%%%%%%%%%%%%%%%%%%%%%%%%%%%%%%%%%%%%%%%%%

% Si se utilizan listings, definir el lenguaje aquí
\providecommand{\myLanguage}{matlab}

% Crear los integrantes del TP con el comando \PutMember donde
%%		1) Apellido, Nombre
%%		2) Número de Padrón
%%		3) E-Mail
\providecommand{\MembersOnCover}[0]
{		
		\PutMember{Anastópulos, Matías}{95120}{matias.anas@gmail.com}
		\PutMember{Gasparovic, Emiliano}{96123}{emilianit2000@gmail.com}
		\PutMember{Manso, Juan} {96133} {juanmanso@gmail.com}
}

\providecommand{\myGroupNumber}{02}


\Pagebreaktrue		% Setea si hay un salto de página en la carátula
\Indextrue
\Siunitxtrue			% Si quiero utilizar el paquete, \siunixtrue. Si no \siunixfalse
\Todonotestrue		% Habilita/Deshabilita las To-Do Notes y las funciones \unsure, \change, \info, \improvement y \thiswillnotshow.
\Listingstrue
\Keywordsfalse
\Putgrouptrue		% Habilita/Deshabilita el \myGroup en los headers
\Videofalse
