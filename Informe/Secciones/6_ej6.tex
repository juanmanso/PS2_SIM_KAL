
	Es importante recalcar, que la ganancia de Kalman $K_k$ es distinta para cada instante $k$. Cuando el sistema es variante en el tiempo, el filtro se adapta a las condiciones de dinámica y ruido instante a instante. Por otro lado cuando el sistema es invariante en el tiempo, puede demostrarse que tanto la matriz $K_k$, como la matriz $P_k$, convergen para tiempo infinito. Es decir:
	
	\begin{equation*}
		K_k \longrightarrow K
	\end{equation*}
	
	\begin{equation*}
		P_k \longrightarrow P
	\end{equation*}
	
	para $k \rightarrow \infty$. Dichas matrices pueden computarse mediante ecuaciones aritméticas discretas de Riccati (DARE). Es decir, $P$ debe cumplir:
	
	\begin{equation*}
		P = A P A^{*} - A P C^{*} (R + C P C^{*})^{-1} C P A + B Q B^{*}
	\end{equation*}
	
	Por otro lado la ganancia de Kalman puede calcularse:
	
	\begin{equation*}
		K = P C^{*} (R + C P C^{*})^{-1}
	\end{equation*}
	
	Dadas estas matrices, surge la posibilidad de implementar una modificación al algoritmo de Kalman, para que en lugar de utilizar las matrices calculadas instante a instante, siempre utilice las matrices a las que va a converger, suponiendo por supuesto que el sistema es invariante en el tiempo. Dicho algoritmo tiene la ventaja de que posee menos procesamiento en tiempo real, dado a que las matrices $K$ y $P$ pueden computarse en tiempo de compilación y guardarse fijas en la memoria. Dicho algoritmo no será óptimo, pero puede demostrarse que la versión modificada convergerá a la versión original con rapidez exponencial, siendo más los beneficios que trae, que los inconvenientes. El objetivo de este punto es comparar las dos variantes y demostrar esto.
