\subsection{Inciso a}

	\graficarEPS{0.6}{bloques}{Diagrama en bloques del modelo.}{fig:bloques}
	Se define la variable de estado asociada a las ecuaciones de movimiento continuo como:
		\begin{equation*}
			\vect{x}(t) = \begin{bmatrix} \vect{p}(t) \\[0.3em] \vect{v}(t) \\[0.3em] \vect{a}(t) \end{bmatrix} \qquad%
			\dot{\vect{x}}(t) = \begin{bmatrix} \dot{\vect{p}}(t) \\[0.3em] \dot{\vect{v}}(t) \\[0.3em] \dot{\vect{a}}(t) \end{bmatrix}
		\end{equation*}

	Así el modelo resulta:
		\begin{equation*}
			\Sigma:
			\begin{cases}
				\dvect{x}(t) = A\: \vect{x}(t) + B \: \vect{\xi}(t) \\
				\vect{y}(t) = C\: \vect{x}(t) + \vect{\eta}(t)
			\end{cases}
		\end{equation*}
	donde $\vect{\xi}(t)$ es el ruido de proceso y $\vect{\eta}(t)$.
	\unsure{Preguntar cuánto tengo que definir y etc.}


\subsection{Inciso b}

	\begin{lstlisting}
config_m;

datos_str = load('datos.mat');

Acel = datos_str.Acel;
Tiempo = datos_str.tiempo;
Pos = datos_str.Pos;
Vel = datos_str.Vel;

dim = 2;			% Se considera sólo x e y
tipos_variables = 3;		% Posición, Velocidad, Aceleración
cant_mediciones = length(Pos);
cant_estados = tipos_variables * dim;



% Datos
var_xip = 3e-4;
var_xiv = 2e-3;
var_xia = 1e-2;

%%%
T = Tiempo(2:end)-Tiempo(1:end-1);	
T = 1;					% Suponiendo equiespaciado

% Variable de estado X = [P;V;A]
I = eye(dim);
Ad =[I		I.*T	(T.^2)/2.*I;
     I*0	I	T.*I;
     I*0	I*0	I;];

% Covarianza del ruido de proceso
Qd = diag([ones(1,dim)*var_xip, ones(1,dim)*var_xiv,ones(1,dim)*var_xia]);
	\end{lstlisting}





	
		
