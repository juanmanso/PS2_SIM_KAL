\subsection{Inciso (a}
	Se define la variable de estado asociada a las ecuaciones de movimiento continuo como:
		\begin{equation*}
			\vect{x}(t) = \begin{bmatrix} \vect{p}(t) \\[0.3em] \vect{v}(t) \\[0.3em] \vect{a}(t) \end{bmatrix} \qquad%
			\dot{\vect{x}}(t) = \begin{bmatrix} \dot{\vect{p}}(t) \\[0.3em] \dot{\vect{v}}(t) \\[0.3em] \dot{\vect{a}}(t) \end{bmatrix}
		\end{equation*}

	Así el modelo resulta:
		\begin{equation*}
			\Sigma:
			\begin{cases}
				\dvect{x}(t) = A\: \vect{x}(t) + B \: \vect{\xi}(t) \\
				\vect{y}(t) = C\: \vect{x}(t) + \vect{\eta}(t)
			\end{cases}
		\end{equation*}
	donde $\vect{\xi}(t)$ es el ruido de proceso y $\vect{\eta}(t)$ el ruido de medición. La matriz $A$ contiene la información de la dinámica del sistema, la matriz $B$ es la matriz de entrada del ruido de proceso, y la matriz $C$ es la matriz de salida, que conecta los estados con las mediciones. En la figura \ref{fig:bloques} puede verse el diagrama en bloques del modelo.
	\graficarEPS{0.6}{bloques}{Diagrama en bloques del modelo.}{fig:bloques}
	
	Para poder tratar el sistema en forma digital, es necesario definir un período de muestreo $T$, que en nuestro caso es igual a la unidad. De esta forma el modelo en tiempo discreto es el siguiente:
	
		\begin{equation*}
			\Sigma_{d}:
			\begin{cases}
				\vect{x_{n + 1}} = A_{d}\: \vect{x_{n}} + B_{d} \: \vect{\xi_{n}} \\
				\vect{y_{n}} = C_{d}\: \vect{x_{n}} + \vect{\eta_{n}}
			\end{cases}
		\end{equation*}
		
	Donde ahora tenemos nuevas matrices para el caso discreto. En nuestro caso, de las ecuaciones de la cinemática se deduce que la matriz de la dinamica del sistema es:

		\begin{equation*}
			A_{d} = \begin{bmatrix} I & IT & I\frac{T^2}{2} \\[0.3em] 0 & I & IT \\[0.3em] 0 & 0 & I \end{bmatrix}
		\end{equation*}

	La matriz de salida $C_{d}$ dependerá de lo que se este midiendo. Por ejemplo para el caso en que midamos todas las variables será:
	
		\begin{equation*}
			C_{d} = \begin{bmatrix} I & 0 & 0 \\[0.3em] 0 & I & 0 \\[0.3em] 0 & 0 & I \end{bmatrix}
		\end{equation*}
		
	Para el caso en que alguna de las variables no se esté midiendo, será necesario eliminar alguna de las tres filas de la matriz. La primera, segunda o tercera, dependiendo de posicion, velocidad y aceleración respectivamente. La matriz B la tomaremos como la identidad en nuestro modelo.
	En cuanto a los ruidos, supondremos que se trata de ruidos blancos de matrices de covarianza $Q_{d}$ para el ruido de proceso y $R_{d}$ para el ruido de medición. Cabe aclarar que el hecho de que se trate de ruidos blancos no implica que $Q_{d}$ y $R_{d}$ sean diagonales. Ambas matrices dan informacion de la correlacion componente a componente de los vectores de ruido.
	
\subsection{Inciso (b}

	El siguiente bloque de código contiene la definición de las matrices $A_{d}$ y $Q_{d}$ que utilizaremos.

	\begin{lstlisting}
	
config_m;

datos_str = load('datos.mat');

Acel = datos_str.Acel;
Tiempo = datos_str.tiempo;
Pos = datos_str.Pos;
Vel = datos_str.Vel;

dim = 2;			% Se considera sólo x e y
tipos_variables = 3;		% Posición, Velocidad, Aceleración
cant_mediciones = length(Pos);
cant_estados = tipos_variables * dim;

% Datos
var_xip = 3e-4;
var_xiv = 2e-3;
var_xia = 1e-2;

%%%
T = Tiempo(2:end)-Tiempo(1:end-1);	
T = 1;					% Suponiendo equiespaciado

% Variable de estado X = [P;V;A]
I = eye(dim);
Ad =[I		I.*T	(T.^2)/2.*I;
     I*0	I	T.*I;
     I*0	I*0	I;];

% Covarianza del ruido de proceso
Qd = diag([ones(1,dim)*var_xip, ones(1,dim)*var_xiv,ones(1,dim)*var_xia]);

	\end{lstlisting}
	
% $Q_d=\begin{bmatrix} \sigma^2_{\xi_{px}} &&&&&\\[0.3em] &\sigma^2_{\xi_{py}}&&&&\\[0.3em] &&\sigma^2_{\xi_{vx}}&&&\\[0.3em] &&&\sigma^2_{\xi_{vy}}&&\\[0.3em] &&&&\sigma^2_{\xi_{ax}}&\\[0.3em] &&&&&\sigma^2_{\xi_{ay}}\end{bmatrix}$






	
		
