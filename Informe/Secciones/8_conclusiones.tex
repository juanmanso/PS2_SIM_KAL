
	\paragraph{}
	Puede sacarse como conclusión de este trabajo que el filtro de Kalman funciona muy bien a la hora de corregir el error introducido por las fuentes de ruido. Ademas es bastante versatil en el sentido de que permite incorporar y resumir información de varias fuentes de mediciones a la vez. Sin embargo hay una fuerte dependencia de los resultados con la observabilidad. Se vió que en muchos casos la misma es un factor limitante importante.
	
	\paragraph{}
	Por otro lado la técnica funciona bien para cuando los ruidos son de media nula. Cuando la media de los ruidos de medición no es nula, y desconocida, la técnica de incluirla como un estado a estimar tiene sus limitaciones, como ya se dijo, ligadas a la observabilidad. De los casos vistos, son pocos en los que se puede estimar las medias en forma satisfactoria.
	
	\paragraph{}
	Se vió también como el sistema funciona bastante bien cuando no se puede hacer una buena estadistica del estado inicial. Cuando no se conoce el estado inicial, o la media del mismo, con certeza, basta con indicarle en la matriz $P$ que el dato es poco confiable para que el algoritmo procese y converja debidamente a la trayectoria.
	
	\paragraph{}
	Se vió también como el algoritmo basado en usar las matrices estacionarias en lugar de calcular la ganancia de Kalman instante a instante funciona razonablemente bien, en comparación con el algoritmo original. Bastó con un corto intervalo de tiempo para que la trayetoria siguiera a la original, y además, el algoritmo modificado posee la misma inmunidad al ruido que el original.
	
	\paragraph{}
	La variante del algoritmo de Kalman con perdida de datos, modificando la matriz $C$, es una herramienta muy util ya que no todo el tiempo se tienen mediciones de todos los sensores. Contar con un algoritmo que haga una sintesis de datos que llegan de varios sensores, en tiempos diferentes, es de suma utilidad práctica y flexibilidad. La posibilidad de poder modificar la matriz $C$ instante a instante esta ligada a otra ventaja del filtro de Kalman poco explorada en el trabajo que es la posibilidad de tratar con sistemas variantes en el tiempo.
	
	\paragraph{}
	En resumen, el filtro de Kalman es de suma importancia práctica dado a que encaja bien en situaciones mas reales que ideales. Esta situaciones son, el hecho de tener que trabajar con sistemas que varian en el tiempo (Aunque en este trabajo no se exploró en profundidad esta ventaja), el hecho de tener que trabajar con señales en presencia de ruido, y en el hecho de tener que sintetizar información de distintos sensores que miden diferentes cosas, y con errores distintos, en un único resultado a la salida del algoritmo.
