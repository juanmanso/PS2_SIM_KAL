
	Se considera un vehículo que se desplaza definiendo una trayectoria tal que la posición en cada instante resulta $\vect{p}(t)$, con una velocidad $\vect{v}(t)$ y una aceleración $\vect{a}(t)$, definidas en un plano de coordenadas $[x,y]$ de acuerdo a:
	\begin{equation*}
		\vect{p}(t) = \begin{bmatrix} p_x(t) \\[0.3em] p_y(t) \end{bmatrix} \qquad%
		\vect{v}(t) = \begin{bmatrix} v_x(t) \\[0.3em] v_y(t) \end{bmatrix} \qquad%
		\vect{a}(t) = \begin{bmatrix} a_x(t) \\[0.3em] a_y(t) \end{bmatrix}% 
	\end{equation*}

	Suponiendo que la dinámica de movimiento satisface las siguientes ecuaciones:
	\begin{equation}
		\begin{cases}
			\vect{\dot{p}}(t) = \vect{v}(t)\\
			\vect{\dot{v}}(t) = \vect{a}(t)\\
			\vect{\dot{a}}(t) = 0
		\end{cases}
		\label{eq:dinamica_enun}
	\end{equation}

