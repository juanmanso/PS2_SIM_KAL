\subsection{Fundamentos}

	Cuando se trabaja con sensores con ruido de media desconocida, no nula y constante, nuestro modelo es el siguiente:

	\begin{equation*}
		\vect{\eta_{n}} = \vect{\tilde{\eta}_{n}} + \vect{\mu}
	\end{equation*}

	\begin{equation*}
		\Sigma_{d}:
		\begin{cases}
			\vect{x_{n + 1}} = A_{d}\: \vect{x_{n}} + B_{d} \: \vect{\xi_{n}} \\
			\vect{y_{n}} = C_{d}\: \vect{x_{n}} + \vect{\tilde{\eta}_{n}} + \vect{\mu}
		\end{cases}
	\end{equation*}

	Donde ahora el nuevo ruido $\vect{\tilde{\eta}_{n}}$ es de media nula, y se representa la media del ruido como $\vect{\mu}$. Una técnica para llevar el nuevo modelo a la forma del modelo anterior consiste en considerar la media del ruido como un estado mas a estimar. De esta forma se incluye dentro de las variables de estado:

	\begin{equation*}
		\vect{z}_{n} = \begin{bmatrix} \vect{x}_{n} \\[0.3em] \vect{\mu}\end{bmatrix} \qquad%
	\end{equation*}
	
	Y el nuevo modelo es:
	
	\begin{equation*}
		\tilde{\Sigma_{d}}:
		\begin{cases}
			\vect{z_{n + 1}} = \tilde{A_{d}}\: \vect{z_{n}} + \tilde{B_{d}} \: \vect{\xi_{n}} \\
			\vect{y_{n}} = \tilde{C_{d}}\: \vect{z_{n}} + \vect{\tilde{\eta}_{n}}
		\end{cases}
	\end{equation*}

	Con las nuevas matrices:
	
	\begin{equation*}
			\tilde{A_{d}} = \begin{bmatrix} A_{d} & 0 \\[0.3em] 0 & I \end{bmatrix}
	\end{equation*}
	
	\begin{equation*}
			\tilde{B_{d}} = \begin{bmatrix} B_{d} \\[0.3em] 0 \end{bmatrix}
	\end{equation*}
	
	\begin{equation*}
			\tilde{C_{d}} = \begin{bmatrix} C_{d} & I \end{bmatrix}
	\end{equation*}
	
	Cabe aclarar que el éxito de esta solución dependerá de la observabilidad de el nuevo sistema. A continuación se expondrán las distintas variantes que pueden darse en cuanto a la observabilidad.

%--------------------------------------------------------------------------------------------------

\subsection{Medición de P - Sesgo en P}

	En la figura \ref{fig:ej4a} se observa el resultado de medir sólo posición, con un sesgo en la misma. Se observa en el gráfico que sin la estimación del sesgo, la estimación de la trayectoria esta desviada de la trayectoria real, de la misma manera que lo estan las mediciones. Cuando se trata de estimar el sesgo, el resultado mejora pero no sigue la trayectoria con la misma efectividad que lo hacia en puntos anteriores, donde no había sesgo, problema ligado a la observabilidad.

	\begin{figure}[H]
		\centering
		\includegraphics[width=1.0\textwidth,keepaspectratio]{Figuras/graf_ej4a.pdf}
		\caption{Estimación De Trayectoria}
		\label{fig:ej4a}
	\end{figure}
	
	En la figura \ref{fig:ej4a_bias} se observa la convergencia de la estimación del sesgo. Se observa que no converge exactamente al valor deseado.
	
	\begin{figure}[H]
		\centering
		\includegraphics[width=0.7\textwidth,keepaspectratio]{Figuras/bias_ej4a.pdf}
		\caption{Estimación Del Sesgo}
		\label{fig:ej4a_bias}
	\end{figure}
	
	En la figura \ref{fig:ej4a_cov} se observa la autocorrelación de las innovaciónes. Se observa que se trata de un proceso blanco.
	
	\begin{figure}[H]
		\centering
		\includegraphics[width=1.0\textwidth,keepaspectratio]{Figuras/covinn_ej4a.pdf}
		\caption{Correlación De Innovaciones}
		\label{fig:ej4a_cov}
	\end{figure}
	
%--------------------------------------------------------------------------------------------------

\subsection{Medición de PVA - Sesgo en P}

	En la figura \ref{fig:ej4b} se observa el resultado de la estimación de la trayectoria. Puede observarse que es similar al caso anterior en cuanto a que funciona mejor que suponer sesgo nulo, sin embargo el seguimiento de la trayectoria no es perfecto, problema nuevamente ligado a la observabilidad.

	\begin{figure}[H]
		\centering
		\includegraphics[width=1.0\textwidth,keepaspectratio]{Figuras/graf_ej4b.pdf}
		\caption{Estimación De Trayectoria}
		\label{fig:ej4b}
	\end{figure}
	
	En la figura \ref{fig:ej4a_bias} se observa la convergencia de la estimación del sesgo. Se observa que no converge exactamente al valor deseado.
	
	\begin{figure}[H]
		\centering
		\includegraphics[width=0.7\textwidth,keepaspectratio]{Figuras/bias_ej4b.pdf}
		\caption{Estimación Del Sesgo}
		\label{fig:ej4b_bias}
	\end{figure}
	
	En la figura \ref{fig:ej4a_cov} se observa la autocorrelación de las innovaciónes. Puede observarse que se trata de un proceso blanco.
	
	\begin{figure}[H]
		\centering
		\includegraphics[width=1.0\textwidth,keepaspectratio]{Figuras/covinn_ej4b.pdf}
		\caption{Correlación De Innovaciones}
		\label{fig:ej4b_cov}
	\end{figure}
	
%--------------------------------------------------------------------------------------------------
	
\subsection{Medición de V - Sesgo en V}

	En la figura \ref{fig:ej4c} se observa el resultado de la estimación. Puede observarse que la estimación no sigue la trayectoria, sino que se aparta cada vez mas de ella a medida que transcurre el tiempo, tanto si se supone el sesgo nulo, como si se trata de estimarlo (Problemas de observabilidad).

	\begin{figure}[H]
		\centering
		\includegraphics[width=1.0\textwidth,keepaspectratio]{Figuras/graf_ej4c.pdf}
		\caption{Estimación De Trayectoria}
		\label{fig:ej4c}
	\end{figure}
	
	En la figura \ref{fig:ej4c_bias} se observa que la estimación converge, pero no a los valores correctos.
	
	\begin{figure}[H]
		\centering
		\includegraphics[width=0.7\textwidth,keepaspectratio]{Figuras/bias_ej4c.pdf}
		\caption{Estimación Del Sesgo}
		\label{fig:ej4c_bias}
	\end{figure}
	
	En la figura \ref{fig:ej4c_cov} se observa la autocorrelación innovaciónes. Puede observarse que el proceso ya no es del todo blanco.
	
	\begin{figure}[H]
		\centering
		\includegraphics[width=1.0\textwidth,keepaspectratio]{Figuras/covinn_ej4c.pdf}
		\caption{Correlación De Innovaciones}
		\label{fig:ej4c_cov}
	\end{figure}
	
%--------------------------------------------------------------------------------------------------

\subsection{Medición de PVA - Sesgo en V}

	En la figura \ref{fig:ej4d} puede observarse el resultado de la estimación. En este caso no tenemos problemas de observabilidad, por lo que la estimación sigue la trayectoria, y además, sigue los valores de velocidad reales.

	\begin{figure}[H]
		\centering
		\includegraphics[width=1.0\textwidth,keepaspectratio]{Figuras/graf_ej4d.pdf}
		\caption{Estimación De Trayectoria}
		\label{fig:ej4d}
	\end{figure}
	
	En la figura \ref{fig:ej4d_bias} se observa la convergencia de la estimación del sesgo. Al no tener problemas de observabilidad, puede observarse que converge a los valores correctos.
	
	\begin{figure}[H]
		\centering
		\includegraphics[width=0.7\textwidth,keepaspectratio]{Figuras/bias_ej4d.pdf}
		\caption{Estimación Del Sesgo}
		\label{fig:ej4d_bias}
	\end{figure}
	
	En la figura \ref{fig:ej4d_cov} se observa la autocorrelación de las innovaciónes. Puede observarse que ya no es un proceso blanco.
	
	\begin{figure}[H]
		\centering
		\includegraphics[width=1.0\textwidth,keepaspectratio]{Figuras/covinn_ej4d.pdf}
		\caption{Correlación De Innovaciones}
		\label{fig:ej4d_cov}
	\end{figure}

%--------------------------------------------------------------------------------------------------

\subsection{Medición de A - Sesgo en A}

	En la figura \ref{fig:ej4e} podemos observar el resultado de la estimación. En este caso no tenemos observabilidad, por lo que el resultado empeora cada vez mas a medida que pasa el tiempo.

	\begin{figure}[H]
		\centering
		\includegraphics[width=1.0\textwidth,keepaspectratio]{Figuras/graf_ej4e.pdf}
		\caption{Estimación De Trayectoria}
		\label{fig:ej4e}
	\end{figure}
	
	En la figura \ref{fig:ej4e_bias} podemos observar la convergencia de la estimación. Si bien la estimación converge, al no tener observabilidad, no lo hace a los valores correctos.
	
	\begin{figure}[H]
		\centering
		\includegraphics[width=0.7\textwidth,keepaspectratio]{Figuras/bias_ej4e.pdf}
		\caption{Estimación Del Sesgo}
		\label{fig:ej4e_bias}
	\end{figure}
	
	En la figura \ref{fig:ej4e_cov} observamos la autocorrelación de las innovaciones. Puede verse que no se trata de un proceso totalmente blanco.
	
	\begin{figure}[H]
		\centering
		\includegraphics[width=1.0\textwidth,keepaspectratio]{Figuras/covinn_ej4e.pdf}
		\caption{Correlación De Innovaciones}
		\label{fig:ej4e_cov}
	\end{figure}

%--------------------------------------------------------------------------------------------------

\subsection{Medición de PVA - Sesgo en A}

	En la figura \ref{fig:ej3f} se observa el resultado de la estimación. Nuevamente no tenemos problemas de observabilidad, por lo que la estimación sigue la trayectoria y, además, sigue la aceleración de forma correcta.

	\begin{figure}[H]
		\centering
		\includegraphics[width=1.0\textwidth,keepaspectratio]{Figuras/graf_ej4f.pdf}
		\caption{Estimación De Trayectoria}
		\label{fig:ej3f}
	\end{figure}
	
	En la figura \ref{fig:ej3f_bias} se observa la convergencia de la estimación del sesgo. Al tener observabilidad, puede observarse que converge a los valores correctos.
	
	\begin{figure}[H]
		\centering
		\includegraphics[width=0.7\textwidth,keepaspectratio]{Figuras/bias_ej4f.pdf}
		\caption{Estimación Del Sesgo}
		\label{fig:ej3f_bias}
	\end{figure}
	
	En la figura \ref{fig:ej3f_cov} se observa la autocorrelación de las innovaciones. Puede verse que no se trata de un proceso blanco.

	\begin{figure}[H]
		\centering
		\includegraphics[width=1.0\textwidth,keepaspectratio]{Figuras/covinn_ej4f.pdf}
		\caption{Correlación De Innovaciones}
		\label{fig:ej3f_cov}
	\end{figure}

%--------------------------------------------------------------------------------------------------
